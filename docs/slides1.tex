% Author: Santiago Faci <santiago.faci@gmail.com>
% http://github.com/sfaci/masc

\documentclass[xcolor={dvipsnames}]{beamer}
\setbeamertemplate{navigation symbols}{}

\usepackage{beamerthemeshadow}
\usepackage[spanish]{babel}
\usepackage{url}
\usepackage[utf8]{inputenc}

\definecolor{resalta}{cmyk}{0,1,0,0}

\usepackage{listings}
\usepackage[T1]{fontenc}
\usepackage{hyperref}
\lstset{basicstyle=\tiny\ttfamily,breaklines=true}
\lstset{framextopmargin=50pt}
\lstset{keywordstyle=\tiny\color{blue}\bfseries}
\lstset{stringstyle=\tiny\color{red}\ttfamily}
\lstset{commentstyle=\tiny\color{OliveGreen}\ttfamily}
\lstset{showstringspaces=false}

\begin{document}
\title{masc: malware (web) scanner}  
\author{Santiago Faci\\ \url{sfaci@fundacionmontessori.com}}
\institute{\small{Colegio Montessori (Zaragoza)}}
\date{\Large{Hackathon Cybercamp 2017}} 

\begin{frame}
\titlepage
\end{frame}

\begin{frame}[plain]\frametitle{Índice}\tableofcontents
\end{frame} 


\section{¿Qué es masc?} 
\begin{frame}\frametitle{¿Qué es masc?} 

    \begin{block}{malware web scanner}
    Escanea un sitio web ya comprometido en busca de malware con el objetivo de eliminarlo
    \end{block}

    \begin{itemize}
        \item Busca y/o limpia, automáticamente, malware en un sitio web que sabemos que ha sido atacado
        \item Aunque no pensemos que hemos sido atacados, podemos ejecutarlo periodicamente para comprobarlo. No todos los ataques son ruidosos
        \item Comportamiento a medida de algunos CMS concretos
        \item Trabaja basándose en diccionarios que pueden ser ampliados y mejorados
        \item La idea es que sea fácilmente extensible para que se adapte a otros CMS y arquitecturas web
    \end{itemize}
\end{frame}

\subsection{¿Quién va a usar masc?}
\begin{frame}\frametitle{¿Quién va a usar masc?}
    \begin{block}{¿Quién va a usar masc?}
        \begin{itemize}
            \item Cualquier que exponga un sitio web a Internet es susceptible de ser atacado
            \item Administradores de sitios web que hayan sido atacados o que quieran comprobar si lo han sido
            \item Para administradores de sistemas puesto que el código se inyecta en las web para utilizar el sistema entero en beneficio propio
        \end{itemize}
    \end{block}
\end{frame}

\subsection{¿Por qué masc?}
\begin{frame}\frametitle{¿Por qué existe masc?}
    \begin{block}{¿Por qué existe masc?}
        \begin{itemize}
            \item Aunque cumplamos todos los protocolos para protegernos, no estamos totalmente a salvo
            \item Precisamente la idea surgió hace un tiempo administrando sitios web que habían sido atacados (la mayoría Wordpress). Para
            entonces tenía unos scripts \emph{bash} que buscaban patrones en el código para eliminar manualmente esos ficheros
            \item Hay muchas herramientas para proteger pero parece díficil encontrarlas para arreglar desastres
            \item Hay infinidad de información sobre cómo limpiar esos sitios web, pero manualmente
            \item Algunas herramientas nunca encontraban nada (linux-malware-detect, ClamAV, . . .)
        \end{itemize}
    \end{block}
\end{frame}

\section{Estado actual}
\begin{frame}\frametitle{Estado actual de masc}
    \begin{block}{Estado actual}
        \begin{itemize}
            \item Ahora mismo \emph{masc} es un comando que escanea un sitio web WordPress en busca de ficheros o contenido sospechosos
            \item Está escrito en Python, lenguaje orientado a objetos, potente, con infinidad de librerías y bastante extendido en el ámbito de la seguridad
            \item Ahora basa su conocimiento en unos diccionarios con aquellos ficheros o contenidos que sabemos que los atacantes generan/emplean
            \item Los diccionarios pueden ampliarse manualmente a través de \emph{masc}
            \item Muestra al usuario los ficheros que parecen sospechosos
        \end{itemize}
    \end{block}
\end{frame}

\section{Propuestas para el hackathon}

\begin{frame}\frametitle{Propuestas para el hackathon}
    \begin{block}{Propuestas para desarrollar}
    \begin{itemize}
        \item Hacer que \emph{masc} trabaje con diferentes CMS (Drupal, Joomla, Magento, . . .)
        \item Convertirlo también en una aplicación web (con ayuda de Django)
        \item Crear una arquitectura que lo haga facilmente extensible
        \item Comprobar los permisos de los ficheros y directorios
        \item Ofrecer al usuario la posibilidad de restaurar el funcionamiento del sitio web
        \item Utilizar las firmas del proyecto \emph{WebMalwareScanner} de \emph{OWASP}
    \end{itemize}
    \end{block}
\end{frame}

\section{Más ideas}

\begin{frame}\frametitle{Ideas para el futuro}
    \begin{block}{Cómo puede crecer masc}
    \begin{itemize}
        \item Ampliar el diccionario de firmas (como hace \emph{WebMalwareScanner})
        \item Mejorar el interfaz web
        \item Escaneo remoto de sitios web
        \item Monitorización en tiempo real para detectar cambios rapidamente y volver a un estado anterior
    \end{itemize}
    \end{block}
\end{frame}

\end{document}
